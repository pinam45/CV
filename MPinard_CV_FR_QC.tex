\PassOptionsToPackage{letterpaper}{geometry}
\documentclass[10pt,sans]{moderncv}
%----------------------------------------------------------------------------------------
%   PACKAGES
%----------------------------------------------------------------------------------------

\usepackage[utf8]{inputenc}
%\usepackage[scale=0.975]{geometry}
\usepackage[top=0.85cm, bottom=0.85cm, left=0.85cm, right=0.85cm]{geometry}
\usepackage{graphicx}
\usepackage[firstyear=2011,lastyear=2018]{moderntimeline}
\usepackage{datenumber,fp}
\usepackage{fontawesome}
%\usepackage{mnsymbol}
%\usepackage{academicons}

%----------------------------------------------------------------------------------------
%   SETTINGS
%----------------------------------------------------------------------------------------

%\makeatletter % changes the catcode of @ to 11
%modify here packages internal commands
%\makeatother % changes the catcode of @ back to 12

\makeatletter

%-------------------------------------------
%   Add preinfo to header
%-------------------------------------------

\newcommand*{\preinfo}[1]{\def\@preinfo{#1}}

% optional maketitle width to force a certain width (if set to 0pt, the width is calculated automatically)
\newlength{\makecvtitlenamewidth}
\setlength{\makecvtitlenamewidth}{0pt}% dummy value
\renewcommand*{\makecvtitle}{%
	{\hypersetup{urlcolor=color2}
	% recompute lengths (in case we are switching from letter to resume, or vice versa)
	\recomputecvlengths%
	% optional detailed information (pre-rendering)
	\def\phonesdetails{}%
	\collectionloop{phones}{% the key holds the phone type (=symbol command prefix), the item holds the number
		\protected@edef\phonesdetails{\phonesdetails\protect\makenewline\csname\collectionloopkey phonesymbol\endcsname\collectionloopitem}}%
	\def\socialsdetails{}%
	\collectionloop{socials}{% the key holds the social type (=symbol command prefix), the item holds the link
		\protected@edef\socialsdetails{\socialsdetails\protect\makenewline\csname\collectionloopkey socialsymbol\endcsname\collectionloopitem}}%
	\newbox{\makecvtitledetailsbox}%
	\savebox{\makecvtitledetailsbox}{%
		\addressfont\color{color2}%
		\begin{tabular}[b]{@{}r@{}}%
			\ifthenelse{\isundefined{\@preinfo}}{}{\makenewline{\@preinfo}}%
			\ifthenelse{\isundefined{\@addressstreet}}{}{\makenewline\addresssymbol\@addressstreet%
			\ifthenelse{\equal{\@addresscity}{}}{}{\makenewline\@addresscity}% if \addresstreet is defined, \addresscity and addresscountry will always be defined but could be empty
			\ifthenelse{\equal{\@addresscountry}{}}{}{\makenewline\@addresscountry}}%
			\phonesdetails% needs to be pre-rendered as loops and tabulars seem to conflict
			\ifthenelse{\isundefined{\@email}}{}{\makenewline\emailsymbol\emaillink{\@email}}%
			\ifthenelse{\isundefined{\@homepage}}{}{\makenewline\homepagesymbol\httplink{\@homepage}}%
			\socialsdetails% needs to be pre-rendered as loops and tabulars seem to conflict
			\ifthenelse{\isundefined{\@extrainfo}}{}{\makenewline\@extrainfo}%
		\end{tabular}
	}%
	% optional photo (pre-rendering)
	\newbox{\makecvtitlepicturebox}%
	\savebox{\makecvtitlepicturebox}{%
		\ifthenelse{\isundefined{\@photo}}%
			{}%
			{%
				\hspace*{\separatorcolumnwidth}%
				\color{color1}%
				\setlength{\fboxrule}{\@photoframewidth}%
				\ifdim\@photoframewidth=0pt%
					\setlength{\fboxsep}{0pt}\fi%
				\framebox{\includegraphics[width=\@photowidth]{\@photo}}%
			}%
	}%
	% name and title
	\newlength{\makecvtitledetailswidth}\settowidth{\makecvtitledetailswidth}{\usebox{\makecvtitledetailsbox}}%
	\newlength{\makecvtitlepicturewidth}\settowidth{\makecvtitlepicturewidth}{\usebox{\makecvtitlepicturebox}}%
	\ifthenelse{\lengthtest{\makecvtitlenamewidth=0pt}}% check for dummy value (equivalent to \ifdim\makecvtitlenamewidth=0pt)
		{\setlength{\makecvtitlenamewidth}{\textwidth-\makecvtitledetailswidth-\makecvtitlepicturewidth}}%
		{}%
	\begin{minipage}[b]{\makecvtitlenamewidth - 10mm}%
		\raggedright
		\titlestyle{\@title}\\
		\hfill\\
		\namestyle{\@firstname\ \@lastname}
	\end{minipage}%
	\hfill%
	% optional detailed information (rendering)
	\llap{\usebox{\makecvtitledetailsbox}}% \llap is used to suppress the width of the box, allowing overlap if the value of makecvtitlenamewidth is forced
	% optional photo (rendering)
	\usebox{\makecvtitlepicturebox}\\[2.5em]%
	% optional quote
	\ifthenelse{\isundefined{\@quote}}%
		{}%
		{{\centering\begin{minipage}{\quotewidth}\vspace*{\quoteDeletedSpace}\centering\quotestyle{\@quote}\end{minipage}\\[2.5em]}}%
	\par% to avoid weird spacing bug at the first section if no blank line is left after \makecvtitle
	}
}

%-------------------------------------------
%   Change tllabelcventry label position
%-------------------------------------------

\renewcommand{\tllabelcventry}[9][color1]{%
\tl@formatendyear{#3}
\tl@formatstartyear{#2}
 \cventry{\tikz[baseline=0pt]{
     \fill [\tl@runningcolor] (0,0)
        rectangle (\hintscolumnwidth,\tl@runningwidth);
     \useasboundingbox (0,-1.5ex)
        rectangle (\hintscolumnwidth,1ex);
     \fill [#1] (0,0)
        ++(\tl@startfraction*\hintscolumnwidth,0pt)
        node [tl@singleyear,anchor=north] {#4}
        rectangle (\tl@endfraction*\hintscolumnwidth,\tl@width-1pt) ;
    \ifissince
       \newdimen\fullcolorwidth
       \pgfmathsetlength\fullcolorwidth{\tl@startfraction*(1+(1-\tl@startfraction)*\tl@nsfrac)*\hintscolumnwidth}
       \shade [left color=#1,right color=#1]
(\tl@startfraction*\hintscolumnwidth,0)
           rectangle (\fullcolorwidth,\tl@width);
       \shade [left color=#1] (\fullcolorwidth,0)
           rectangle (\tl@endfraction*\hintscolumnwidth,\tl@width);
     \else
        \fill [#1] (\tl@startfraction*\hintscolumnwidth,0)
            rectangle (\tl@endfraction*\hintscolumnwidth,\tl@width);
     \fi
     }
}
{#5}{#6}{#7}{#8}{#9}%
}

\makeatother

%-------------------------------------------
%   Theme related colors
%-------------------------------------------

% color0      main default color, normally left to black
% color1      primary scheme color
% color2      secondary scheme color

%-------------------------------------------
%   Theme font and style redefinition
%-------------------------------------------

\AtBeginDocument{%

% fonts
\renewcommand*{\titlefont}{\fontsize{32}{14}\mdseries\upshape}
\renewcommand*{\namefont}{\fontsize{19}{14}\mdseries\upshape}
%\renewcommand*{\addressfont}{\normalsize\mdseries\upshape}
%\renewcommand*{\quotefont}{\large\slshape}
%\renewcommand*{\sectionfont}{\Large\bfseries\upshape}
%\renewcommand*{\subsectionfont}{\large\upshape\fontseries{sb}\selectfont}
%\renewcommand*{\hintfont}{\bfseries}

% styles
\renewcommand*{\titlestyle}[1]{{\titlefont\textcolor{color1}{#1}}}
\renewcommand*{\namestyle}[1]{{\namefont\textcolor{color2!85}{#1}}}
%\renewcommand*{\addressstyle}[1]{{\addressfont\textcolor{color2}{#1}}}
%\renewcommand*{\quotestyle}[1]{{\quotefont\textcolor{color1}{#1}}}
%\renewcommand*{\sectionstyle}[1]{{\sectionfont\textcolor{color1}{#1}}}
%\renewcommand*{\subsectionstyle}[1]{{\subsectionfont\textcolor{color1}{#1}}}
%\renewcommand*{\hintstyle}[1]{{\hintfont\textcolor{color0}{#1}}}

}

%-------------------------------------------
%   hyperref colors
%-------------------------------------------

\AtBeginDocument{%

\hypersetup{
	colorlinks=true,
	%linkcolor=red,
	%anchorcolor=black,
	citecolor=green,
	%filecolor=cyan,
	%menucolor=red,
	urlcolor=color1,
	linktoc=all
}

}

%-------------------------------------------
%   Symbols from FontAwesome
%-------------------------------------------

\AtBeginDocument{%

\renewcommand*{\addresssymbol}       {\faMapMarker\ }
\renewcommand*{\mobilephonesymbol}   {\faMobile\ }
\renewcommand*{\fixedphonesymbol}    {\foPhone\ }
\renewcommand*{\faxphonesymbol}      {\faFax\ }
\renewcommand*{\emailsymbol}         {\faEnvelopeO\ }
\renewcommand*{\homepagesymbol}      {\faGlobe\ }
\renewcommand*{\linkedinsocialsymbol}{\faLinkedin\ }
\renewcommand*{\twittersocialsymbol} {\faTwitter\ }
\renewcommand*{\githubsocialsymbol}  {\faGithub\ }

}

%-------------------------------------------
%   Commands definitions
%-------------------------------------------

% Used to indicate the level in the languages section
\newcommand{\cvskill}[2]{%
\textcolor{color2}{\textbf{#1}}
\foreach \x in {1,...,5}{%
	\space{\ifnumgreater{\x}{#2}{\color{color1!30}}{\color{color1}}\faCircle}}%
}

\newcommand{\cvitemcom}[2]{%
	\cvitemwithcomment{}{\listitemsymbol #1}{#2}%
}

% Colored href
\newcommand{\Chref}[3][blue]{\href{#2}{\color{#1}{#3}}}%

% Space to delete between the sections
\newcommand{\deletedSpace}{0mm}

% Space to delete before the quote
\newcommand{\quoteDeletedSpace}{0mm}

%-------------------------------------------
%   Age computation
%-------------------------------------------

\newcounter{birthday}
\newcounter{today}
\setmydatenumber{birthday}{1996}{06}{19}
\setmydatenumber{today}{\the\year}{\the\month}{\the\day}
\FPsub\result{\thetoday}{\thebirthday}
\FPdiv\myage{\result}{365.2425}
\FPtrunc\myage{\myage}{0}

%-------------------------------------------
%   ModernCV theme
%-------------------------------------------

\moderncvtheme[blue]{banking}
%\moderncvtheme[blue]{classic}
%\renewcommand{\familydefault}{\sfdefault}
\nopagenumbers{}


% Space to delete between the sections
\renewcommand{\deletedSpace}{-.5mm}

% Space to delete before the quote
\renewcommand{\quoteDeletedSpace}{-.5mm}

%----------------------------------------------------------------------------------------
%   PERSONAL DATA
%----------------------------------------------------------------------------------------

\preinfo{%
	%\faUser\ \myage{} ans, célibataire
}
\firstname{Maxime}
\familyname{Pinard}
\title{Étudiant en informatique}
\address{Saguenay, Québec, Canada}{}
\phone[mobile]{+33~687~925~509}
%\phone[fixed]{}
%\phone[fax]{}
\email{maxime.pin@live.fr}
\homepage{maxime.pinard.info}
\social[linkedin]{maxime-pinard}
\social[github]{pinam45}
%\extrainfo{%
% extra info
%}
%\photo[70pt]{MPinard.jpg}
\quote{En dernière année d'école d'ingénieurs et de maîtrise en informatique, je recherche un stage de 6 mois qui commencerais en août 2019.}


%----------------------------------------------------------------------------------------
%   DOCUMENT BODY
%----------------------------------------------------------------------------------------

\begin{document}
	\maketitle

	%-------------------------------------------
	%   Course
	%-------------------------------------------

	\section{Cursus}
		\tlcventry{2018/9}{0}
			{Maîtrise en informatique}
			{Université du Québec à Chicoutimi}
			{Saguenay, Québec, Canada}
			{}
			{\textit{Actuellement en dernière année, en double diplôme avec l'UTBM}}
		\tlcventry{2016/9}{0}
			{Diplôme d'ingénieur en informatique, spécialité imagerie}
			{Université de Technologie de Belfort-Montbéliard}
			{Belfort, France}
			{\textit{eq. Maîtrise}}
			{\textit{Dernière année réalisée en double diplôme à l'UQAC}}
		\tlcventry{2014/9}{2016/8}
			{DEUTEC}
			{Université de Technologie de Belfort-Montbéliard}
			{Sèvenans, France}
			{\textit{eq. Baccalauréat}}
			{Classe préparatoire avant choix de la filière diplômante}
		\tlcventry{2011/9}{2014/8}
			{Baccalauréat S, option SVT, spécialité Mathématiques}
			{Lycée Louis Aragon}
			{Héricourt, France}
			{\textit{Mention Bien, eq. DEC}}
			{}

	%-------------------------------------------
	%   Languages
	%-------------------------------------------

	\vspace*{\deletedSpace}
	\section{Langues}
		\cvdoubleitem{\bfseries Français}{langue maternelle}{\bfseries Anglais}{B2 (BULATS 68), usage professionnel}
		\cvdoubleitem{\bfseries Espagnol}{niveau scolaire}{\bfseries Japonais}{élémentaire}

	%-------------------------------------------
	%   Computer skills
	%-------------------------------------------

	\vspace*{\deletedSpace}
	\section{Compétences informatiques}
		\cvline{Langages}{C++, C, Java, \LaTeX{}/Ti\textit{k}Z, Python, Bash, CUDA, SQL, HTML/CSS, Assembleur, VBA\ldots}
		\cvline{Méthodes/Outils}{Agile, Test unitaires, Patrons de conception, UML, BPMN, git, CMake, Make, Maven, Regex}
		\cvline{Métaheuristiques}{Recherche tabou, recuit simulé, algorithmes génétiques, algorithmes de colonies de fourmis}
		\cvline{Bibliothèques}{boost, fmt, spdlog, OpenGL, OpenCV, OpenMP, MPI, GTest, Catch2, Capstone, Keystone, Json, SFML, ImGui, GLFW, GLM, libmpg123, JavaFx, JUnit4}
		\cvitem{Administration système}{GNU/Linux (Debian 8), hébergement de sites web Java EE et de services web Docker}
		\cvline{IDE}{Visual Studio, JetBrains IntelliJ IDEA et CLion, Eclipse}
		%\cvline{Logiciels}{MathLab, Unity (+ package Vuforia AR et SteamVR pour HTC Vive)}

	%-------------------------------------------
	%   Work experience
	%-------------------------------------------

	\vspace*{\deletedSpace}
	\section{Expériences Professionnelles}
		\tllabelcventry{2018/7}{2018/8}{07/2018 - 08/2018}
			{Auxiliaire ambulancier}
			{Ambulances Phoenix, 2 mois}
			{Héricourt, France}
			{Prise en charge et transport de patients}
			{}
		\tllabelcventry{2017/8}{2018/1}{08/2017 - 01/2018}
			{Stagiaire}
			{Direction Générale de l'Armement Maîtrise de l'Information, 6 mois}
			{Bruz, France}
			{}
			{}
			\vspace{-5pt}
			\begin{itemize}
				\item Travail sur GenDbg, un débogueur multi langages / OS / architectures:\\
					\phantom{=}Développement en C du module de désassemblage pour les architectures MIPS et des tests unitaires associés
				\item Travail sur YaCo, plugin IDA Pro de rétro-ingénieurie collaborative utilisant Git:\\
					\phantom{=}Portage en C++ et amélioration de la gestion du dépôt Git et de la capture des évènements IDA
			\end{itemize}
			\vspace{5pt}
		\tllabelcventry{2016/2}{2016/2}{02/2016}
			{Animateur}
			{JAB France, 1 semaine}
			{Evolène, Suisse}
			{Camp de ski de 40 jeunes, équipe de 15 animateurs}
			{}
		\tllabelcventry{2015/2}{2015/2}{02/2015}
			{Animateur}
			{JAB France, 1 semaine}
			{Contamines, France}
			{Camp de ski de 50 jeunes, équipe de 20 animateurs}
			{}
		\tllabelcventry{2015/1}{2015/2}{01/2015}
			{Stagiaire}
			{Souchier SAS, 4 semaines}
			{Héricourt, France}
			{Jointage et montage d'appareils de désenfumage}
			{}

	%-------------------------------------------
	\newpage
	%-------------------------------------------

	%-------------------------------------------
	%   Others qualifications
	%-------------------------------------------

	%\vspace*{\deletedSpace}
	\section{Certifications autres}
		\cvlistitem{Permis de conduire}{}
		\cvlistitem{Prévention et Secours Civiques de niveau 1 (PSC1)}{}

	%-------------------------------------------
	%   Projects
	%-------------------------------------------

	\vspace*{\deletedSpace}
	\section{Projets}
		\subsection{Personnels}
			\cvitemcom{Lecteur de musique [C++, ImGui, SFML, spdlog, libmpg123]}
				{\href{https://github.com/pinam45/MagicPlayer}{MagicPlayer}}
			\cvitemcom{Convertisseur de base, exemple ImGui/SFML [C++, ImGui, SFML]}
				{\href{https://github.com/pinam45/BaseConverter}{BaseConverter}}
			\cvitemcom{Librairie graphique en console pour systèmes basé Unix et Windows [C]}
				{\href{https://github.com/pinam45/ConsoleControl}{ConsoleControl}}
			\cvitemcom{Dungeon crawler avec niveaux à génération procédurale [Java, JavaFx]}
				{\href{https://github.com/TiWinDeTea/Raoul-the-Game}{Raoul-the-Game}}
			\cvitemcom{Implémentation \LaTeX{}/Ti\textit{k}Z des 1er et 4e de couverture pour rapports de stage UTBM}
				{\href{https://github.com/pinam45/utbm-latex-internship-report-covers}{utbm-latex-internship-report-covers}}
			\cvitemcom{Implémentation \LaTeX{}/Ti\textit{k}Z Beamer du theme pour présentations UTBM}
				{\href{https://github.com/pinam45/utbm-beamer-theme}{utbm-beamer-theme}}
			\cvitemcom{Jeu type Snake multijoueur (réseau local) [C++, SFML]}
				{\href{https://github.com/TiWinDeTea/PapraGame}{PapraGame}}
		\subsection{Recherche}
			\cvitemcom{Calculateur d'hyperplans dans des géométries finies de dimension 4 (orienté performances) [C++]}
				{\href{https://github.com/Lomadriel/HyperplaneFinder}{HyperplaneFinder}}
			\cvitem{}{~~~> \textit{publication \cite{veldkamp}}}
		\subsection{Stage}
			\cvitemcom{Plugin Hex-Rays IDA de reverse-engineering collaboratif [C++, GTest, IDA]}
				{\href{https://github.com/DGA-MI-SSI/YaCo}{YaCo}, \href{https://github.com/pinam45/UTBM_ST40_Rapport_de_stage_DGA}{Rapport de stage}}
			\cvitemcom{Module d'assemblage/désassemblage pour code assembleur MIPS [C, Capstone, Keystone, GTest]}
				{\href{https://github.com/pinam45/UTBM_ST40_Rapport_de_stage_DGA}{Rapport de stage}}
		\subsection{École}
			\cvitemcom{Construction et rendu d'objet paramétrique avec OpenGL [C++, GLFW, ImGui]}
				{\href{https://github.com/pinam45/UTBM\_IN55\_ParametricObjectsConstruction}{ParametricObjectsConstruction}}
			\cvitemcom{Clone de MiniMetro: simulateur de gestion de métros [Java, JavaFx]}
				{\href{https://github.com/TiWinDeTea/MagicMetro}{MagicMetro}}
			\cvitemcom{Jeux de Pogo avec IA MinMax/AlphaBeta [C++, ConsoleControl]}
				{\href{https://github.com/pinam45/UTBM_IA41_Pogo}{Pogo}}
			\cvlistitem{Serveur de stockage de fichiers multi-utilisateurs [C++, SFML]}

	%-------------------------------------------
	%   Publications
	%-------------------------------------------

	\vspace*{\deletedSpace}
	\begin{thebibliography}{}
		%\itemsep=-2pt
		\bibitem{veldkamp}
		[1] \href{https://arxiv.org/abs/1806.08965}{Veldkamp Spaces of Low-Dimensional Ternary Segre Varieties}, [[publication à venir]] (2018)\\
		\textit{Metod Saniga, Jérôme Boulmier, Maxime Pinard, Frédéric Holweck}
	\end{thebibliography}

	%-------------------------------------------
	%   Interests
	%-------------------------------------------

	\vspace*{\deletedSpace}
	\section{Centres d'intérêts}
		\cvline{L'informatique}
			{Les nouveaux paradigmes de programmation, la génération procédurale, la recherche opérationnelle, l'optimisation, la cryptographie et l'évolution de l'informatique quantique.}
		\cvline{Les sciences}
			{Les mathématiques, simulation de système physique, physique quantique\ldots}
		\cvline{Le sport}
			{Le ski et les sports d'hivers, le vélo (VTT en club pendant 3 ans), la marche en montagne.}
\end{document}
