\documentclass[10pt,a4paper,sans]{moderncv}
%----------------------------------------------------------------------------------------
%   PACKAGES
%----------------------------------------------------------------------------------------

\usepackage[utf8]{inputenc}
%\usepackage[scale=0.975]{geometry}
\usepackage[top=0.85cm, bottom=0.85cm, left=0.85cm, right=0.85cm]{geometry}
\usepackage{graphicx}
\usepackage[firstyear=2011,lastyear=2018]{moderntimeline}
\usepackage{datenumber,fp}
\usepackage{fontawesome}
%\usepackage{mnsymbol}
%\usepackage{academicons}

%----------------------------------------------------------------------------------------
%   SETTINGS
%----------------------------------------------------------------------------------------

%\makeatletter % changes the catcode of @ to 11
%modify here packages internal commands
%\makeatother % changes the catcode of @ back to 12

\makeatletter

%-------------------------------------------
%   Add preinfo to header
%-------------------------------------------

\newcommand*{\preinfo}[1]{\def\@preinfo{#1}}

% optional maketitle width to force a certain width (if set to 0pt, the width is calculated automatically)
\newlength{\makecvtitlenamewidth}
\setlength{\makecvtitlenamewidth}{0pt}% dummy value
\renewcommand*{\makecvtitle}{%
	{\hypersetup{urlcolor=color2}
	% recompute lengths (in case we are switching from letter to resume, or vice versa)
	\recomputecvlengths%
	% optional detailed information (pre-rendering)
	\def\phonesdetails{}%
	\collectionloop{phones}{% the key holds the phone type (=symbol command prefix), the item holds the number
		\protected@edef\phonesdetails{\phonesdetails\protect\makenewline\csname\collectionloopkey phonesymbol\endcsname\collectionloopitem}}%
	\def\socialsdetails{}%
	\collectionloop{socials}{% the key holds the social type (=symbol command prefix), the item holds the link
		\protected@edef\socialsdetails{\socialsdetails\protect\makenewline\csname\collectionloopkey socialsymbol\endcsname\collectionloopitem}}%
	\newbox{\makecvtitledetailsbox}%
	\savebox{\makecvtitledetailsbox}{%
		\addressfont\color{color2}%
		\begin{tabular}[b]{@{}r@{}}%
			\ifthenelse{\isundefined{\@preinfo}}{}{\makenewline{\@preinfo}}%
			\ifthenelse{\isundefined{\@addressstreet}}{}{\makenewline\addresssymbol\@addressstreet%
			\ifthenelse{\equal{\@addresscity}{}}{}{\makenewline\@addresscity}% if \addresstreet is defined, \addresscity and addresscountry will always be defined but could be empty
			\ifthenelse{\equal{\@addresscountry}{}}{}{\makenewline\@addresscountry}}%
			\phonesdetails% needs to be pre-rendered as loops and tabulars seem to conflict
			\ifthenelse{\isundefined{\@email}}{}{\makenewline\emailsymbol\emaillink{\@email}}%
			\ifthenelse{\isundefined{\@homepage}}{}{\makenewline\homepagesymbol\httplink{\@homepage}}%
			\socialsdetails% needs to be pre-rendered as loops and tabulars seem to conflict
			\ifthenelse{\isundefined{\@extrainfo}}{}{\makenewline\@extrainfo}%
		\end{tabular}
	}%
	% optional photo (pre-rendering)
	\newbox{\makecvtitlepicturebox}%
	\savebox{\makecvtitlepicturebox}{%
		\ifthenelse{\isundefined{\@photo}}%
			{}%
			{%
				\hspace*{\separatorcolumnwidth}%
				\color{color1}%
				\setlength{\fboxrule}{\@photoframewidth}%
				\ifdim\@photoframewidth=0pt%
					\setlength{\fboxsep}{0pt}\fi%
				\framebox{\includegraphics[width=\@photowidth]{\@photo}}%
			}%
	}%
	% name and title
	\newlength{\makecvtitledetailswidth}\settowidth{\makecvtitledetailswidth}{\usebox{\makecvtitledetailsbox}}%
	\newlength{\makecvtitlepicturewidth}\settowidth{\makecvtitlepicturewidth}{\usebox{\makecvtitlepicturebox}}%
	\ifthenelse{\lengthtest{\makecvtitlenamewidth=0pt}}% check for dummy value (equivalent to \ifdim\makecvtitlenamewidth=0pt)
		{\setlength{\makecvtitlenamewidth}{\textwidth-\makecvtitledetailswidth-\makecvtitlepicturewidth}}%
		{}%
	\begin{minipage}[b]{\makecvtitlenamewidth - 10mm}%
		\raggedright
		\titlestyle{\@title}\\
		\hfill\\
		\namestyle{\@firstname\ \@lastname}
	\end{minipage}%
	\hfill%
	% optional detailed information (rendering)
	\llap{\usebox{\makecvtitledetailsbox}}% \llap is used to suppress the width of the box, allowing overlap if the value of makecvtitlenamewidth is forced
	% optional photo (rendering)
	\usebox{\makecvtitlepicturebox}\\[2.5em]%
	% optional quote
	\ifthenelse{\isundefined{\@quote}}%
		{}%
		{{\centering\begin{minipage}{\quotewidth}\vspace*{\quoteDeletedSpace}\centering\quotestyle{\@quote}\end{minipage}\\[2.5em]}}%
	\par% to avoid weird spacing bug at the first section if no blank line is left after \makecvtitle
	}
}

%-------------------------------------------
%   Change tllabelcventry label position
%-------------------------------------------

\renewcommand{\tllabelcventry}[9][color1]{%
\tl@formatendyear{#3}
\tl@formatstartyear{#2}
 \cventry{\tikz[baseline=0pt]{
     \fill [\tl@runningcolor] (0,0)
        rectangle (\hintscolumnwidth,\tl@runningwidth);
     \useasboundingbox (0,-1.5ex)
        rectangle (\hintscolumnwidth,1ex);
     \fill [#1] (0,0)
        ++(\tl@startfraction*\hintscolumnwidth,0pt)
        node [tl@singleyear,anchor=north] {#4}
        rectangle (\tl@endfraction*\hintscolumnwidth,\tl@width-1pt) ;
    \ifissince
       \newdimen\fullcolorwidth
       \pgfmathsetlength\fullcolorwidth{\tl@startfraction*(1+(1-\tl@startfraction)*\tl@nsfrac)*\hintscolumnwidth}
       \shade [left color=#1,right color=#1]
(\tl@startfraction*\hintscolumnwidth,0)
           rectangle (\fullcolorwidth,\tl@width);
       \shade [left color=#1] (\fullcolorwidth,0)
           rectangle (\tl@endfraction*\hintscolumnwidth,\tl@width);
     \else
        \fill [#1] (\tl@startfraction*\hintscolumnwidth,0)
            rectangle (\tl@endfraction*\hintscolumnwidth,\tl@width);
     \fi
     }
}
{#5}{#6}{#7}{#8}{#9}%
}

\makeatother

%-------------------------------------------
%   Theme related colors
%-------------------------------------------

% color0      main default color, normally left to black
% color1      primary scheme color
% color2      secondary scheme color

%-------------------------------------------
%   Theme font and style redefinition
%-------------------------------------------

\AtBeginDocument{%

% fonts
\renewcommand*{\titlefont}{\fontsize{32}{14}\mdseries\upshape}
\renewcommand*{\namefont}{\fontsize{19}{14}\mdseries\upshape}
%\renewcommand*{\addressfont}{\normalsize\mdseries\upshape}
%\renewcommand*{\quotefont}{\large\slshape}
%\renewcommand*{\sectionfont}{\Large\bfseries\upshape}
%\renewcommand*{\subsectionfont}{\large\upshape\fontseries{sb}\selectfont}
%\renewcommand*{\hintfont}{\bfseries}

% styles
\renewcommand*{\titlestyle}[1]{{\titlefont\textcolor{color1}{#1}}}
\renewcommand*{\namestyle}[1]{{\namefont\textcolor{color2!85}{#1}}}
%\renewcommand*{\addressstyle}[1]{{\addressfont\textcolor{color2}{#1}}}
%\renewcommand*{\quotestyle}[1]{{\quotefont\textcolor{color1}{#1}}}
%\renewcommand*{\sectionstyle}[1]{{\sectionfont\textcolor{color1}{#1}}}
%\renewcommand*{\subsectionstyle}[1]{{\subsectionfont\textcolor{color1}{#1}}}
%\renewcommand*{\hintstyle}[1]{{\hintfont\textcolor{color0}{#1}}}

}

%-------------------------------------------
%   hyperref colors
%-------------------------------------------

\AtBeginDocument{%

\hypersetup{
	colorlinks=true,
	%linkcolor=red,
	%anchorcolor=black,
	citecolor=green,
	%filecolor=cyan,
	%menucolor=red,
	urlcolor=color1,
	linktoc=all
}

}

%-------------------------------------------
%   Symbols from FontAwesome
%-------------------------------------------

\AtBeginDocument{%

\renewcommand*{\addresssymbol}       {\faMapMarker\ }
\renewcommand*{\mobilephonesymbol}   {\faMobile\ }
\renewcommand*{\fixedphonesymbol}    {\foPhone\ }
\renewcommand*{\faxphonesymbol}      {\faFax\ }
\renewcommand*{\emailsymbol}         {\faEnvelopeO\ }
\renewcommand*{\homepagesymbol}      {\faGlobe\ }
\renewcommand*{\linkedinsocialsymbol}{\faLinkedin\ }
\renewcommand*{\twittersocialsymbol} {\faTwitter\ }
\renewcommand*{\githubsocialsymbol}  {\faGithub\ }

}

%-------------------------------------------
%   Commands definitions
%-------------------------------------------

% Used to indicate the level in the languages section
\newcommand{\cvskill}[2]{%
\textcolor{color2}{\textbf{#1}}
\foreach \x in {1,...,5}{%
	\space{\ifnumgreater{\x}{#2}{\color{color1!30}}{\color{color1}}\faCircle}}%
}

\newcommand{\cvitemcom}[2]{%
	\cvitemwithcomment{}{\listitemsymbol #1}{#2}%
}

% Colored href
\newcommand{\Chref}[3][blue]{\href{#2}{\color{#1}{#3}}}%

% Space to delete between the sections
\newcommand{\deletedSpace}{0mm}

% Space to delete before the quote
\newcommand{\quoteDeletedSpace}{0mm}

%-------------------------------------------
%   Age computation
%-------------------------------------------

\newcounter{birthday}
\newcounter{today}
\setmydatenumber{birthday}{1996}{06}{19}
\setmydatenumber{today}{\the\year}{\the\month}{\the\day}
\FPsub\result{\thetoday}{\thebirthday}
\FPdiv\myage{\result}{365.2425}
\FPtrunc\myage{\myage}{0}

%-------------------------------------------
%   ModernCV theme
%-------------------------------------------

\moderncvtheme[blue]{banking}
%\moderncvtheme[blue]{classic}
%\renewcommand{\familydefault}{\sfdefault}
\nopagenumbers{}


%----------------------------------------------------------------------------------------
%   PERSONAL DATA
%----------------------------------------------------------------------------------------

\preinfo{%
	\faUser\ \myage{} ans, célibataire
}
\firstname{Maxime}
\familyname{Pinard}
\title{Stagiaire développement logiciel}
\address{Héricourt, France}{}
\phone[mobile]{+33~687~925~509}
%\phone[fixed]{}
%\phone[fax]{}
\email{maxime.pin@live.fr}
\homepage{maxime.pinard.info}
\social[linkedin]{maxime-pinard}
\social[github]{pinam45}
%\extrainfo{%
% extra info
%}
\photo[70pt]{MPinard.jpg}
\quote{Élève-ingénieur en informatique actuellement en 1ère année, je cherche un stage {\normalsize(BAC+4)} de développement de 6 mois pour septembre 2017}


%----------------------------------------------------------------------------------------
%   DOCUMENT BODY
%----------------------------------------------------------------------------------------

\begin{document}
	\maketitle
	\vspace*{-15mm}

	%-------------------------------------------
	%   Course
	%-------------------------------------------

	\section{Cursus}
		\tlcventry{2016}{0}{Diplôme d'ingénieur informatique, spécialité imagerie}{Université de Technologie de Belfort-Montbéliard}{Belfort, France}{\textit{Bac+5 (eq. Master degree)}}
		%{{\fontsize{12}{0}\textcolor{color1}{$\filledmedtriangleright$}} Actuellement en 3ème année, \textit{Bac+3 (eq. Bachelor degree)}}
		{Actuellement en 3ème année, \textit{Bac+3 (eq. Bachelor degree)}}
		\tlcventry{2014}{2016}{DEUTEC}{Université de Technologie de Belfort-Montbéliard}{Sèvenans, France}{\textit{Bac+2}}{Classe préparatoire avant choix de la filière diplômante}
		\tlcventry{2011}{2014}{Baccalauréat S, option SVT, spécialité Mathématiques}{Lycée Louis Aragon}{Héricourt, France}{\small\textit{Mention Bien (eq. DEC)}}{}

	%-------------------------------------------
	%   Languages
	%-------------------------------------------

	\vspace*{\deletedSpace}
	\section{Langues}
		\cvdoubleitem{\bfseries Français}{langue maternelle}{\bfseries Anglais}{B2 (BULATS 68), usage professionnel}
		\cvdoubleitem{\bfseries Espagnol}{niveau scolaire}{\bfseries Japonais}{élémentaire}

	%-------------------------------------------
	%   Computer skills
	%-------------------------------------------

	\vspace*{\deletedSpace}
	\section{Compétences informatiques}
		\cvline{Langages}{C++, C, Java 8 (SE et EE), C\#, \LaTeX, SARL, Bash, CUDA, SQL, HTML/CSS, Assembleur, VBA\ldots}
		\cvline{Outils}{UML, git, Maven, CMake, Make}
		\cvitem{Administration système}{GNU/Linux (Debian 8), serveur personnel pour hébergement web Java EE (Tomcat 8)}
		\cvline{Bibliothèques}{Capstone, Keystone, Qt, SFML, ImGui, JavaFx, JUnit4}
		\cvline{IDE}{Visual Studio, JetBrains IntelliJ IDEA et CLion, Eclipse}
		\cvline{Logiciels}{MathLab, Unity (+ package Vuforia AR et SteamVR pour HTC Vive)}

	%-------------------------------------------
	%   Projects
	%-------------------------------------------

	\vspace*{\deletedSpace}
	\section{Projets}
		\cvlistitem{Captation des mouvements d'une personne et transmission réseau vers un avatar en réalité augmentée avec Unity (scripting en C\#), Kinect et Vuforia}
		\cvitemwithcomment{}{\listitemsymbol Simulateur de gestion de métros en Java avec JavaFx (clone de MiniMetro)}{\Chref[color1]{https://github.com/TiWinDeTea/MagicMetro}{MagicMetro}}
		\cvitemwithcomment{}{\listitemsymbol Jeu de la vie de Conway en SARL (langage orienté agents)}{\Chref[color1]{https://github.com/TiWinDeTea/sarl-game-of-life}{sarl game of life}}
		\cvitemwithcomment{}{\listitemsymbol Dungeon crawler en Java avec JavaFx}{\Chref[color1]{https://github.com/TiWinDeTea/Raoul-the-Game}{Raoul the Game}}
		\cvitemwithcomment{}{\listitemsymbol Snake like multijoueur (réseau local) en C++ avec SFML}{\Chref[color1]{https://github.com/TiWinDeTea/PapraGame}{PapraGame}}
		\cvlistitem{Serveur de stockage de fichiers multi-utilisateurs en C++ avec SFML}

	%-------------------------------------------
	%   Work experience
	%-------------------------------------------

	\vspace*{\deletedSpace}
	\section{Expériences Professionnelles}
		\tllabelcventry{2016/2}{2016/2}{Févr 2016}{Animateur}{JAB France, 1 semaine}{Evolène, Suisse}{Camp de ski de 40 jeunes, équipe de 15 animateurs}{}
		\vspace*{-1.8mm}
		\tllabelcventry{2015/2}{2015/3}{Janv--Févr 2015}{Animateur}{JAB France, 1 semaine}{Contamines, France}{Camp de ski de 50 jeunes, équipe de 20 animateurs}{}
		\vspace*{-1.8mm}
		\tllabelcventry{2015/1}{2015/2}{Janv--Févr 2015}{Stagiaire}{Souchier SAS, 4 semaines}{Héricourt, France}{Jointage et montage d'appareils de désenfumage}{}

	%-------------------------------------------
	%   Others qualifications
	%-------------------------------------------

	\vspace*{\deletedSpace}
	\section{Certifications autres}
		\cvlistitem{Permis de conduire français}{}
		\cvlistitem{Prévention et Secours Civiques de niveau 1 (PSC1)}{}

	%-------------------------------------------
	%   Interests
	%-------------------------------------------

	\vspace*{\deletedSpace}
	\section{Centres d'intérêts}
		\cvline{L'informatique}{Les nouveaux paradigmes de programmation, la génération procédurale, l'intelligence artificielle, la cryptographie\ldots{}, leur mise en pratique, et l'évolution de l'informatique quantique.}
		\cvline{Les sciences}{Les mathématiques, simulation de système physique, physique quantique\ldots}
		\cvline{Le sport}{Le ski et les sports d'hivers, le vélo (VTT en club pendant 3 ans), la marche en montagne}

\end{document}
